\documentclass{article}

\usepackage{booktabs}
\usepackage{tabularx}
\usepackage{hyperref}

\title{Development Plan\\\progname}

\author{\authname}

\date{}

%% Comments

\usepackage{color}

\newif\ifcomments\commentstrue %displays comments
%\newif\ifcomments\commentsfalse %so that comments do not display

\ifcomments
\newcommand{\authornote}[3]{\textcolor{#1}{[#3 ---#2]}}
\newcommand{\todo}[1]{\textcolor{red}{[TODO: #1]}}
\else
\newcommand{\authornote}[3]{}
\newcommand{\todo}[1]{}
\fi

\newcommand{\wss}[1]{\authornote{magenta}{SS}{#1}} 
\newcommand{\plt}[1]{\authornote{cyan}{TPLT}{#1}} %For explanation of the template
\newcommand{\an}[1]{\authornote{cyan}{Author}{#1}}

%% Common Parts

\newcommand{\progname}{Software Engineering} 
\newcommand{\authname}{Team \#6, Six Sense
\\ Omar Alam
\\ Sathurshan Arulmohan
\\ Nirmal Chaudhari
\\ Kalp Shah
\\ Jay Sharma
}        

\usepackage{hyperref}
    \hypersetup{colorlinks=true, linkcolor=blue, citecolor=blue, filecolor=blue,
                urlcolor=blue, unicode=false}
    \urlstyle{same}
                                


\begin{document}

\maketitle

\begin{table}[hp]
\caption{Revision History} \label{TblRevisionHistory}
\begin{tabularx}{\textwidth}{llX}
\toprule
\textbf{Date} & \textbf{Developer(s)} & \textbf{Change}\\
\midrule
2025-09-22 & Nirmal, Sathurshan, Omar, Kalp, Jay & Initial Write-up\\
\bottomrule
\end{tabularx}
\end{table}

\newpage{}

This document describes the development plan for the spatial awareness device \textit{audio360}. It contains internal logistics that our team will adhere to, and measures in place to ensure the confidentiality of our device.

\section{Confidential Information?}

There is no confidential information to protect.

\section{IP to Protect}

There is no IP to protect.

\section{Copyright License}

Our team will be adopting the MIT \href{https://github.com/Nirmal-code/SixSense/blob/main/LICENSE}{License}.

\section{Team Meeting Plan}

\wss{How often will you meet? where?}

\wss{If the meeting is a physical location (not virtual), out of an abundance of
caution for safety reasons you shouldn't put the location online}

\wss{How often will you meet with your industry advisor?  when?  where?}

\wss{Will meetings be virtual?  At least some meetings should likely be
in-person.}

\wss{How will the meetings be structured?  There should be a chair for all meetings.  There should be an agenda for all meetings.}

\section{Team Communication Plan}

\wss{Issues on GitHub should be part of your communication plan.}

\section{Team Member Roles}

\begin{itemize}
  \item Leader (Sathurshan Arulmohan)
    \begin{itemize}
      \item Point of contact for professor and TAs. 
      \item Ensures that the team is on track to meet deadlines.
      \item Project planning and scheduling.
    \end{itemize}
  \item Meeting chair (Kalp Shah)
    \begin{itemize}
      \item Creates the agenda for meetings.
      \item Facilitates the meetings.
      \item Ensures that the team sticks to the agenda.
    \end{itemize}
  \item Reviewer (Jay Sharma)
    \begin{itemize}
      \item Reviews all deliverables before deadline to ensure all sections are completed.
      \item Reach out to the reviewers of each section to ensure they approve it before the deadline.
    \end{itemize}
  \item Note taker (Nirmal Chaudhari)
    \begin{itemize}
      \item Takes notes during meetings.
      \item Create meeting notes summary after the meeting ends to grasp meeting content.
      \item Coordinates with project leader after meeting ends for action items. 
    \end{itemize}
  \item System Specialist (Omar Alam)
    \begin{itemize}
      \item Manages and validates system design at a high level. 
      \item Responsible for hardware \& software interfaces.
      \item Handles budgeting and expenses, ensures total budget remains under \$500.
    \end{itemize}
\end{itemize}

\section{Workflow Plan}

\begin{itemize}
	\item How will you be using git, including branches, pull request, etc.?
	\item How will you be managing issues, including template issues, issue
	classification, etc.?
  \item Use of CI/CD
\end{itemize}

\section{Project Decomposition and Scheduling}

\begin{itemize}
  \item How will you be using GitHub projects?
  \item Include a link to your GitHub project
\end{itemize}

\wss{How will the project be scheduled?  This is the big picture schedule, not
details. You will need to reproduce information that is in the course outline
for deadlines.}

\section{Proof of Concept Demonstration Plan}

\wss{What is the main risk, or risks, for the success of your project?  What will you
demonstrate during your proof of concept demonstration to convince yourself that
you will be able to overcome this risk?}

\hspace{1cm}

The biggest risk for the success of the project is the risk of not having a synchronized microphone array that can recognize the directionality of an audio source.
This is the fundamental technology that the rest of the project and strech goals are build upon (excluding the audio classification pipeline).

The goal of the proof of concept demonstration is to showcase the functionality of a microphone array on a controlled environment - a table top lab surface.
The microphone array should be synchronized and capable of recognizing the directionality of a single audio source (sine frequency sound) with a maximum error of 45 degrees on the 2D plane of the table.
Though this demonstration doesn't showcase the ability of recognizing multiple audio sources with the directional microphone array on a smart glasses frame (which is the final goal), it showcases the feasibility of recoginizing the directionality of an audio source anywhere on a 2D plane with adequete accuracy.
If successful, this would give us confidence to extend the project with the \hyperlink{audio_classification_pipeline}{audio classification pipeline}, development of recognizing more than one audio source, extending the audio analysis pipeline onto the smart glasses, and audio filtering on direction. 

\section{Expected Technology}

\wss{What programming language or languages do you expect to use?  What external
libraries?  What frameworks?  What technologies.  Are there major components of
the implementation that you expect you will implement, despite the existence of
libraries that provide the required functionality.  For projects with machine
learning, will you use pre-trained models, or be training your own model?  }

\wss{The implementation decisions can, and likely will, change over the course
of the project.  The initial documentation should be written in an abstract way;
it should be agnostic of the implementation choices, unless the implementation
choices are project constraints.  However, recording our initial thoughts on
implementation helps understand the challenge level and feasibility of a
project.  It may also help with early identification of areas where project
members will need to augment their training.}

Topics to discuss include the following:

\begin{itemize}
\item Specific programming language
\item Specific libraries
\item Pre-trained models
\item Specific linter tool (if appropriate)
\item Specific unit testing framework
\item Investigation of code coverage measuring tools
\item Specific plans for Continuous Integration (CI), or an explanation that CI
  is not being done
\item Specific performance measuring tools (like Valgrind), if
  appropriate
\item Tools you will likely be using?
\end{itemize}

\wss{git, GitHub and GitHub projects should be part of your technology.}

\section{Coding Standard}

\wss{What coding standard will you adopt?}

\newpage{}

\section*{Appendix --- Reflection}

\wss{Not required for CAS 741}

The purpose of reflection questions is to give you a chance to assess your own
learning and that of your group as a whole, and to find ways to improve in the
future. Reflection is an important part of the learning process.  Reflection is
also an essential component of a successful software development process.  

Reflections are most interesting and useful when they're honest, even if the
stories they tell are imperfect. You will be marked based on your depth of
thought and analysis, and not based on the content of the reflections
themselves. Thus, for full marks we encourage you to answer openly and honestly
and to avoid simply writing ``what you think the evaluator wants to hear.''

Please answer the following questions.  Some questions can be answered on the
team level, but where appropriate, each team member should write their own
response:


\begin{enumerate}
    \item Why is it important to create a development plan prior to starting the
    project?

    \textbf{Omar Alam:} Creating a development plan for a large project is crucial to ensure that the team is aligned on goals
    and that expectations are understood by everyone. It helps keep everyone accountable since we are all agreeing to the same rules
    and any disagreements on processes can be resolved by referring back to the plan.
    \item In your opinion, what are the advantages and disadvantages of using
    CI/CD?

    \textbf{Omar Alam}: CI/CD has many advantages, most of which tie back to the concept of iterative development. It allows us to ensure
    that the code branch that is considered production-ready ("main") is always in a deployable state. It also allows us to catch bugs when multiple 
    developers are working on and merging in the same codebase. The main disadvantage of CI/CD is the intiial setup time and feasability of using it.
    In some cases, CI/CD may not be feasible which might end up applying to our project. This is mainly due to the embedded nature of our project 
    and the lack of support for CI/CD on embedded systems. Furthermore, setting up unit tests in CI/CD will require us to write platform agnostic
    code that can also run on the embedded micro-controller and our development machines. This will require extra effort and guidelines to get right.
    \item What disagreements did your group have in this deliverable, if any,
    and how did you resolve them?

    \textbf{Omar Alam}: Since the team project is exepcted to require embedded hardware, we had disagreements on whether or not we could adequately
    specify elements needed to complete the "Expected Technology" section. We resolved this by researching the hardware we might end up using and 
    determining that we could make reasonable assumptions on the technology we would be using.
\end{enumerate}

\newpage{}

\section*{Appendix --- Glossary}


\hypertarget{audio_classification_pipeline}{\underline{Audio Classification Pipeline}}: 
Software pipeline aiming to classify the sound sources.
For example, the audio can be recognized as a car, person speaking, alarm, etc.

\newpage{}

\section*{Appendix --- Team Charter}

\subsection*{External Goals}

The team's primary goal is to gain expertise in embedded software development while contributing a novel technique to the broader scientific and engieering community.
The main focus is that each member strengthens or acquires technical skills that will be directly valuable in their future careers.
As a stetch goal, the team aims to rank within the top five projects at the Capstone Expo, showcasing both technical innovation and professional presentation.

\subsection*{Attendance}

\subsubsection*{Expectations}

\wss{What are your team's expectations regarding meeting attendance (being on
time, leaving early, missing meetings, etc.)?}

\subsubsection*{Acceptable Excuse}

\wss{What constitutes an acceptable excuse for missing a meeting or a deadline?
What types of excuses will not be considered acceptable?}

\subsubsection*{In Case of Emergency}

\wss{What process will team members follow if they have an emergency and cannot
attend a team meeting or complete their individual work promised for a team
deliverable?}

\subsection*{Accountability and Teamwork}

\subsubsection*{Quality} 

Our team expects all members to come to meetings fully prepared with completed work, 
clear progress updates, and constructive contributions to discussions.
All deliverables must meet our established quality standards including proper code review, testing, and thorough documentation. 
We emphasize proactive communication, early problem identification, and mutual support to ensure the team's success and maintain high 
professional standards throughout the project.

\subsubsection*{Attitude}

We expect all team members to maintain a positive, collaborative attitude and treat each other with respect and professionalism. 
We encourage open communication, constructive feedback, and diverse perspectives while ensuring all interactions remain productive and inclusive. 
Our team adopts a conflict resolution approach that prioritizes direct communication, active listening, and finding mutually beneficial solutions to 
maintain a supportive working environment.

\subsubsection*{Stay on Track}

We will track team performance through regular check-ins, commit frequency, and meeting attendance to ensure everyone contributes their fair share. 
For members who exceed expectations and consistently deliver quality work, we will reward them with treats like Tim Hortons coffee or donuts to recognize their efforts. 
For those not contributing their fair share, we will initially cover for them and provide support, but after repeated instances, we will issue formal warnings and escalate 
to our TA or instructor if necessary to maintain team accountability and project success.

\subsubsection*{Team Building}

We will build team cohesion by hosting social gatherings at our off-campus house, where we can work together on projects in a relaxed environment and spend time hanging out
to strengthen our relationships. These informal settings will help us bond as a team while still maintaining productivity and collaboration on our project goals.

\subsubsection*{Decision Making} 

We will make decisions through consensus when possible, seeking common ground and agreement among all team members. However, when there is a clear divide or disagreement that 
cannot be resolved through discussion, we will hold a formal vote to reach a decision and move forward with the project. This approach ensures we respect everyone's input while 
maintaining progress when consensus cannot be achieved.

\end{document}