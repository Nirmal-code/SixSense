\documentclass{article}

\usepackage{booktabs}
\usepackage{tabularx}
\usepackage{hyperref}
\usepackage{longtable}
\usepackage{array}
\usepackage{geometry}
\usepackage{float}
\usepackage{pdflscape}

\hypersetup{
    colorlinks=true,       % false: boxed links; true: colored links
    linkcolor=red,          % color of internal links (change box color with linkbordercolor)
    citecolor=green,        % color of links to bibliography
    filecolor=magenta,      % color of file links
    urlcolor=cyan           % color of external links
}

\title{Hazard Analysis\\\progname}

\author{\authname}

\date{}

%% Comments

\usepackage{color}

\newif\ifcomments\commentstrue %displays comments
%\newif\ifcomments\commentsfalse %so that comments do not display

\ifcomments
\newcommand{\authornote}[3]{\textcolor{#1}{[#3 ---#2]}}
\newcommand{\todo}[1]{\textcolor{red}{[TODO: #1]}}
\else
\newcommand{\authornote}[3]{}
\newcommand{\todo}[1]{}
\fi

\newcommand{\wss}[1]{\authornote{magenta}{SS}{#1}} 
\newcommand{\plt}[1]{\authornote{cyan}{TPLT}{#1}} %For explanation of the template
\newcommand{\an}[1]{\authornote{cyan}{Author}{#1}}

%% Common Parts

\newcommand{\progname}{Software Engineering} 
\newcommand{\authname}{Team \#6, Six Sense
\\ Omar Alam
\\ Sathurshan Arulmohan
\\ Nirmal Chaudhari
\\ Kalp Shah
\\ Jay Sharma
}        

\usepackage{hyperref}
    \hypersetup{colorlinks=true, linkcolor=blue, citecolor=blue, filecolor=blue,
                urlcolor=blue, unicode=false}
    \urlstyle{same}
                                



\def\changemargin#1#2{\list{}{\rightmargin#2\leftmargin#1}\item[]}
\let\endchangemargin=\endlist

\newcommand{\cellrule}{\par\vspace{-0.3em}\noindent\rule{\linewidth}{0.2pt}\par}


\begin{document}

\maketitle
\thispagestyle{empty}

~\newpage

\pagenumbering{roman}

\begin{table}[hp]
\caption{Revision History} \label{TblRevisionHistory}
\begin{tabularx}{\textwidth}{llX}
\toprule
\textbf{Date} & \textbf{Developer(s)} & \textbf{Change}\\
\midrule
Date1 & Name(s) & Description of changes\\
Date2 & Name(s) & Description of changes\\
... & ... & ...\\
\bottomrule
\end{tabularx}
\end{table}

~\newpage

\tableofcontents

~\newpage

\pagenumbering{arabic}

\wss{You are free to modify this template.}

\section{Introduction}

A hazard is anything that prevents the Audio360 system from notifying users with important sounds near them with high precision and accuracy. For deaf and hard-of-hearing individuals who rely on the system for situational awareness, any failure to detect, classify, or display audio information could result in missed safety cues, social interactions, or environmental awareness.

This hazard analysis identifies potential failure modes in the Audio360 audio localization system and establishes safety requirements to ensure reliable operation in real-world scenarios.

\section{Scope and Purpose of Hazard Analysis}

The scope of this document is to identify possible hazards within the Audio360 system components, the effects and causes of failures, mitigation steps, and resulting safety and security requirements.

Potential losses that could be incurred due to system failures include physical injury from missed emergency vehicle warnings or approaching machinery, household accidents from undetected safety alerts, missed social interactions and communication opportunities, reduced independence and confidence in daily activities, and loss of user trust in the assistive technology system.

\section{System Boundaries and Components}

This section is broken down into two subsections: one for components within the
system boundary, and one for components outside the system boundary.

\subsection{Inside the System Boundary}

The following components are within the system's control and responsibility:

\begin{itemize}
\item \textbf{Embedded firmware:} Real-time operating system and all
embedded software running on the processing unit.

\item \textbf{Signal processing module:} Real-time digital signal processing
algorithms including frequency domain transforms, filtering, and time-domain
analysis.

\item \textbf{Direction of arrival (DoA) estimation:} Algorithms for
computing sound source direction on a 2D plane based on time difference of
arrival and phase differences across microphones.

\item \textbf{Audio classification engine:} Sound fingerprinting and
classification logic to identify and categorize detected sounds (e.g.,
speech, vehicles, alarms).

\item \textbf{Visualization Controller:} \label{comp:viz_controller}
Component reposonsible for creating and sending visualization output to the
glasses.

\item \textbf{Configuration and calibration:} Microphone array calibration
routines and system configuration parameters.
\end{itemize}

\subsection{Outside the System Boundary}

The following external entities interact with the system but are not under
its direct control:

\begin{itemize}
\item \textbf{Users:} Individuals who are deaf or hard of hearing wearing
the device. Their actions, responses to alerts, and interpretation of
displayed information are outside the system's control.

\item \textbf{Environmental sounds:} Acoustic signals in the physical
environment, including speech, vehicle noises, alarms, and ambient sounds.
The system detects and processes these but does not generate or control
them.

\item \textbf{Audio capture subsystem:} Synchronized sampling logic for the
microphone array, including analog-to-digital conversion interfaces and
buffer management. This component is included in the microcontroller and is not
within our system's control.

\item \textbf{Physical microphone hardware:} Microphone sensors that capture
acoustic pressure waves. While the system controls their digital interface,
the physical transduction mechanism is external.

\item \textbf{Smart glasses hardware:} The physical display device,
including its screen, optics, power management, and form factor. The system
sends display commands but does not control the hardware's internal
operation.

\item \textbf{\href{def:microcontroller}{Microcontroller}:}
\label{comp:microcontroller} Component responsible for processing real time
data of sensor inputs. This hardware component's performance and reliability
are outside our system's control.

\item \textbf{Power supply:} Battery or external power source providing
electrical power to system components. Power management at the hardware
level is external to the software system.

\item \textbf{Physical environment:} Room acoustics, ambient noise levels,
temperature, and other environmental factors that affect sound propagation
and microphone performance.
\end{itemize}



\section{Critical Assumptions}

\wss{These assumptions that are made about the software or system.  You should
minimize the number of assumptions that remove potential hazards.  For instance,
you could assume a part will never fail, but it is generally better to include
this potential failure mode.}

\section{Failure Mode and Effect Analysis}

\subsection{Severity Mapping Table}
The severity, occurrence, and detection ratings used in the FMEA table are defined as follows:
\begin{table}[H]
\begin{tabular}{c|c|c|c}

    Rating & Severity & Occurrence & Detection \\\hline
    1 & Negligible & Rare & Always detected \\\hline
    2 & Minor & Uncommon & Easy to detect \\ \hline
    3 & Major & Occasional & Moderately difficult to detect \\\hline
    4 & Critical & Frequent & Difficult to detect \\\hline
    5 & Catastrophic & Very frequent & Impossible to detect \\


\end{tabular}
\end{table}

\subsection{Priority Mapping Table}

The priority level is determined by the summation of the severity, occurrence,
and detection ratings:

\begin{table}[H]
    \begin{tabular}{c|c}
        Product of severities & Priority Level \\\hline
        1 - 4 & Low Priority \\\hline
        5 - 10 & Medium Priority \\\hline
        10 - 15 & High Priority \\

    \end{tabular}
\end{table}

\newpage
\pdfpagewidth=15in
\pdfpageheight=11.5in
\newgeometry{left=1.5cm, right=1cm, top=1cm, bottom=1cm}

    
    \begin{longtable}{
        >{\raggedright\arraybackslash}p{0.5cm}%     (\#)
        >{\raggedright\arraybackslash}p{3.0cm}%   (Component)
        >{\raggedright\arraybackslash}p{3.0cm}%   (Potential Failure)
        >{\raggedright\arraybackslash}p{3.0cm}%   (Effect on User)
        >{\raggedright\arraybackslash}p{3.0cm}%   (Likely Cause)
        >{\centering\arraybackslash}p{1.5cm}%   (Severity)
        >{\centering\arraybackslash}p{2.4cm}%     (Occurence Frequency)
        >{\raggedright\arraybackslash}p{3.0cm}%     (Detection Method)
        >{\centering\arraybackslash}p{2.0cm}%     (Detection Likelihood)
        >{\centering\arraybackslash}p{2.4cm}%   (Priority)
        >{\raggedright\arraybackslash}p{3.0cm}%     (Recommended Mitigation)
        >{\raggedright\arraybackslash}p{3.0cm}%
      }

    \toprule
    \textbf{\#} & 
    \textbf{Component / Function} & 
    \textbf{Potential Failure Mode} &
    \textbf{Effect on User / System}&
    \textbf{Likely Cause(s)} & 
    \textbf{Severity} &
    \textbf{Occurrence Frequency} &
    \textbf{Detection Method} &
    \textbf{Detection Likelihood} &
    \textbf{Priority \hspace{15pt}(S + O + D)} &
    \textbf{Recommended Action} &
    \textbf{Relevant Requirement(s)}\\
    \midrule
    \endfirsthead
    \toprule
    
    \midrule
    \endhead


    1 & Microphone &  Unresponsive / Distorted Microphone. & Not able
    to effectively localize audio. \cellrule Unable to provide warnings to user. & 
    Microphone circuit failure. \cellrule Microphone damage. \cellrule Excessive ambient noise. &
    5 & 1  &
    Multiple corrupted microphone data frames. \cellrule Excessive white noise detection.
    \cellrule Short circuit detection for microphones. & 3 & 9 & Detect failure
    using microphone audio (or lack thereof) and notify user. & FR1.4, FR2.3, FR3.5, FR7.2, NFR2.1  \\

    \midrule
    2 & Visualization Controller & Disconnection from display. & Unable to provide 
    visual notifications to user. & Depending on connection type: cable, 
    wireless interference, dropped connection. & 3 & 2 &
    Loss of connection signal. \cellrule Failure to send data to display. & 1 & 6 &
    There are no safety requirements that can mitigate this hazard. & NFR2.1\\

    \midrule


    3 & Embedded Firmware & System crashes and freezes. & The system remains unusable
    until it is restarted. \cellrule Unable to provide notifications to user. &
    Software bugs. \cellrule Insufficient Error Handling \cellrule Insufficient
    Requirements. & 5 & 3 & System watchdog timer 
    \cellrule Halt Interrupt. & 1 & 9 & Implement a watchdog timer to reset the
    system in the event of a crash. \cellrule Implement adequate logging to
    diagnose the cause of firmware failure during post-mortem. \cellrule Robust testing and 
    error handling. & FR1.3, FR4.1, FR4.4 \\
    \midrule

    4 & Sound Detection (Audio360 Engine)& Failure to detect important sounds. & Failure to notify
    user of critical sounds. & Insufficient microphone quality. \cellrule Poor 
    classification algorithm. \cellrule Excessive ambient noise. & 4 & 3 &
    This is difficult to detect without extensive testing. \cellrule Impossible
    to detect in production. & 5 & 12 & Have a thorough
    library of sounds to detect and test against. \cellrule Measure microphone 
    performance on sound library. & FR3.5, FR4.4, FR5.1, FR6.1  \\

    \midrule

    5 & Sound Classifier (Audio360 Engine) & Misclassification of sounds. & Notify the user of the 
    wrong sounds. & Insufficient classification algorithm \cellrule Insufficient
    microphone quality. & 3 & 2 & This is difficult to detect
    without extensive user testing.\cellrule Impossible to detect in production. & 5 & 10 & Extensive testing in real-world
    environments and simulation. \cellrule Verify microphone quality during operation.
    & FR4.4, FR5.1, FR5.4, NFR5.1, NFR5.2 \\


    \midrule

    6 & Sound Localizer (Audio360 Engine) & Inaccurate direction determination for sounds. & Misinform
    the user of the direction of important sounds. & Incorrect localization algorithm.
    \cellrule Insufficient microphone quality. & 4 & 2 & This is
    difficult to detect without extensive localization testing. \cellrule Impossible to detect
    in production. & 5 & 11 & Extensive localization testing in real-world environments and simulation.
    & FR5.2, NFR5.3 \\
    
    \bottomrule
    \end{longtable}

\restoregeometry

\pdfpagewidth=8.5in
\pdfpageheight=11in

\subsection{Out of Scope Failure Modes}
The project involves off the shelf hardware components with their own possible
failure modes and possible mitigations. Since the hardware and electrical 
components are not being designed as part of the project, 
these failure modes are considered out of scope:
\begin{itemize}
    \item Battery failure / damage.
    \item Physical display damage.
    \item Power supply issues.
    \item Physical damage to the glasses.
\end{itemize}


\section{Safety and Security Requirements}

\wss{Newly discovered requirements.  These should also be added to the SRS.  (A
rationale design process how and why to fake it.)}

\section{Roadmap}

\wss{Which safety requirements will be implemented as part of the capstone timeline?
Which requirements will be implemented in the future?}

\newpage{}

\section*{Appendix --- Reflection}

\begin{enumerate}
    \item What went well while writing this deliverable? 
    
    \textbf{Sathurshan:} The team had a strong understanding of the
    safety-critical nature of the system. Some members of the team also had
    experience in writing a Hazard Analysis document from extra-curricular
    activities. This made performing the hazard analysis more straightforward,
    as we were able to effectively identify and evaluate potential risks within
    the system.

    \item What pain points did you experience during this deliverable, and how
    did you resolve them?

    \textbf{Sathurshan:} Some sections of this deliverable were dependent on the
    completion of other sections within this document and the SRS. As a result,
    managing the timeline for these sections was challenging. To resolve this,
    we coordinated closely with the owners of the dependent sections, which
    improved collaboration and allowed us to exchange constructive feedback to
    ensure consistency across the document.
    
    \item Which of your listed risks had your team thought of before this
    deliverable, and which did you think of while doing this deliverable? For
    the latter ones (ones you thought of while doing the Hazard Analysis), how
    did they come about?

    \textbf{Sathurshan:} Before this deliverable, I had already identified the
    risk of microphone failures, as it serves as the system's primary input.
    During the hazard analysis, we also identified the risk of unauthorized
    access to system components, particularly data stored on the microcontroller
    and modification of the software. This risk emerged as we considered
    non-physical hazards, which the following questions allude to.

    \item Other than the risk of physical harm (some projects may not have any
    appreciable risks of this form), list at least 2 other types of risk in
    software products. Why are they important to consider?

    \textbf{Sathurshan:} Other forms of software-related risks include data
    privacy violations and security vulnerabilities that allow unauthorized
    access without user consent. These are critical to consider because they can
    user data can be used against them, resulting in ethical consequences.

\end{enumerate}

\end{document}