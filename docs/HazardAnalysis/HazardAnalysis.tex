\documentclass{article}

\usepackage{booktabs}
\usepackage{tabularx}
\usepackage{hyperref}
\usepackage{longtable}
\usepackage{array}
\usepackage{geometry}
\usepackage{float}
\usepackage{pdflscape}
\usepackage{xr-hyper}
\usepackage{cite}

\hypersetup{
    colorlinks=true,       % false: boxed links; true: colored links
    linkcolor=red,          % color of internal links (change box color with linkbordercolor)
    citecolor=green,        % color of links to bibliography
    filecolor=magenta,      % color of file links
    urlcolor=cyan           % color of external links
}

\title{Hazard Analysis\\\progname}

\author{\authname}

\date{}

%% Comments

\usepackage{color}

\newif\ifcomments\commentstrue %displays comments
%\newif\ifcomments\commentsfalse %so that comments do not display

\ifcomments
\newcommand{\authornote}[3]{\textcolor{#1}{[#3 ---#2]}}
\newcommand{\todo}[1]{\textcolor{red}{[TODO: #1]}}
\else
\newcommand{\authornote}[3]{}
\newcommand{\todo}[1]{}
\fi

\newcommand{\wss}[1]{\authornote{magenta}{SS}{#1}} 
\newcommand{\plt}[1]{\authornote{cyan}{TPLT}{#1}} %For explanation of the template
\newcommand{\an}[1]{\authornote{cyan}{Author}{#1}}

%% Common Parts

\newcommand{\progname}{Software Engineering} 
\newcommand{\authname}{Team \#6, Six Sense
\\ Omar Alam
\\ Sathurshan Arulmohan
\\ Nirmal Chaudhari
\\ Kalp Shah
\\ Jay Sharma
}        

\usepackage{hyperref}
    \hypersetup{colorlinks=true, linkcolor=blue, citecolor=blue, filecolor=blue,
                urlcolor=blue, unicode=false}
    \urlstyle{same}
                                



\def\changemargin#1#2{\list{}{\rightmargin#2\leftmargin#1}\item[]}
\let\endchangemargin=\endlist

\newcommand{\cellrule}{\par\vspace{-0.3em}\noindent\rule{\linewidth}{0.2pt}\par}

\setlength{\parindent}{0pt}

\begin{document}

\maketitle
\thispagestyle{empty}

~\newpage

\pagenumbering{roman}

\begin{table}[hp]
\caption{Revision History} \label{TblRevisionHistory}
\begin{tabularx}{\textwidth}{llX}
\toprule
\textbf{Date} & \textbf{Developer(s)} & \textbf{Change}\\
\midrule
2025-10-06 &  Nirmal, Sathurshan, Omar, Kalp, Jay & Initial Write-up\\
\bottomrule
\end{tabularx}
\end{table}

~\newpage

\tableofcontents

~\newpage

\pagenumbering{arabic}

\section{Introduction}

A hazard is anything that prevents the Audio360 system from notifying users with
important sounds near them with high precision and accuracy. For deaf and
hard-of-hearing individuals who rely on the system for situational awareness,
any failure to detect, classify, or display audio information could result in
missed safety cues, social interactions, or degraded environmental awareness.

This hazard analysis identifies potential failure modes in the Audio360 audio 
localization system and establishes safety requirements to ensure reliable
operation in real-world scenarios.

\section{Scope and Purpose of Hazard Analysis}

The scope of this document is to identify possible hazards within the Audio360
system components, the effects and causes of failures, mitigation steps, and
resulting safety and security requirements.

Potential losses that could be incurred due to system failures include physical
injury from missed emergency vehicle warnings or approaching machinery,
household accidents from undetected safety alerts, missed social interactions
and communication opportunities, reduced independence and confidence in daily
activities, and loss of user trust in the assistive technology system.

\section{System Boundaries and Components}

This section is broken down into two subsections: one for components within the
system boundary, and one for components outside the system boundary.

\subsection{Inside the System Boundary}

The following components are within the system's control and responsibility:

\begin{itemize}
\item \textbf{Embedded firmware:} Real-time operating system and all
embedded software running on the processing unit.

\item \textbf{Signal processing module:} Real-time digital signal processing
algorithms including frequency domain transforms, filtering, and time-domain
analysis.

\item \textbf{Direction of arrival (DoA) estimation:} Algorithms for
computing sound source direction on a 2D plane based on time difference of
arrival and phase differences across microphones.

\item \textbf{Audio classification engine:} Sound fingerprinting and
classification logic to identify and categorize detected sounds (e.g.,
speech, vehicles, alarms).

\item \textbf{Visualization Controller:} \label{comp:viz_controller}
Component responsible for creating and sending visualization output to the
glasses.

\item \textbf{Configuration and calibration:} Microphone array calibration
routines and system configuration parameters.
\end{itemize}

\subsection{Outside the System Boundary}

The following external entities interact with the system but are not under
its direct control:

\begin{itemize}
\item \textbf{Users:} Individuals who are deaf or hard of hearing wearing
the device. Their actions, responses to alerts, and interpretation of
displayed information are outside the system's control.

\item \textbf{Environmental sounds:} Acoustic signals in the physical
environment, including speech, vehicle noises, alarms, and ambient sounds.
The system detects and processes these but does not generate or control
them.

\item \textbf{Audio capture subsystem:} Synchronized sampling logic for the
microphone array, including analog-to-digital conversion interfaces and
buffer management. This component is included in the microcontroller and is not
within our system's control.

\item \textbf{Physical microphone hardware:} Microphone sensors that capture
acoustic pressure waves. While the system controls their digital interface,
the physical transduction mechanism is external.

\item \textbf{Smart glasses hardware:} The physical display device,
including its screen, optics, power management, and form factor. The system
sends display commands but does not control the hardware's internal
operation.

\item \textbf{Microcontroller:}
\label{comp:microcontroller} Component responsible for processing real time
data of sensor inputs. This hardware component's performance and reliability
are outside our system's control.

\item \textbf{Power supply:} Battery or external power source providing
electrical power to system components. Power management at the hardware
level is external to the software system.

\item \textbf{Physical environment:} Room acoustics, ambient noise levels,
temperature, and other environmental factors that affect sound propagation
and microphone performance.
\end{itemize}



\section{Critical Assumptions}

\begin{enumerate}
    \item Microphone
    \begin{itemize}
        \item All microphones are able to capture frequencies between at least
        20 Hz to 20kHz, which is the human audio frequency range.
        \cite{Neuroscience2001}.
        \item Microphones operate correctly within normal operating temperatures
        of 0-35°C. 
    \end{itemize}
    
    \item Microcontroller
    \begin{itemize}
        \item The analog-to-digital converters (ADC) that comes with the
        hardware provides stable conversions in real-time.
        \item Processor's clock frequency remains stable under normal
        operating temperatures 0-35°C.
    \end{itemize}

    \item Output Display
    \begin{itemize}
        \item Visual indicators on the smart glasses remain visible in typical
        lighting conditions. 
    \end{itemize}

    \item System Integration
    \begin{itemize}
        \item Voltage and power is maintained by the controller firmware itself
        and we do not need to manage that in software. 
    \end{itemize}
    
\end{enumerate}

\section{Failure Mode and Effect Analysis}

\subsection{Severity Mapping Table}
The severity, occurrence, and detection ratings used in the FMEA table are defined as follows:
\begin{table}[H]
\begin{tabular}{c|c|c|c}

    Rating & Severity & Occurrence & Detection \\\hline
    1 & Negligible & Rare & Always detected \\\hline
    2 & Minor & Uncommon & Easy to detect \\ \hline
    3 & Major & Occasional & Moderately difficult to detect \\\hline
    4 & Critical & Frequent & Difficult to detect \\\hline
    5 & Catastrophic & Very frequent & Impossible to detect \\


\end{tabular}
\end{table}

\subsection{Priority Mapping Table}

The priority level is determined by the summation of the severity, occurrence,
and detection ratings:

\begin{table}[H]
    \begin{tabular}{c|c}
        Product of severities & Priority Level \\\hline
        1 - 4 & Low Priority \\\hline
        5 - 10 & Medium Priority \\\hline
        10 - 15 & High Priority \\

    \end{tabular}
\end{table}

\newpage
\pdfpagewidth=15in
\pdfpageheight=11.5in
\newgeometry{left=1.5cm, right=1cm, top=1cm, bottom=1cm}

    
    \begin{longtable}{
        >{\raggedright\arraybackslash}p{0.5cm}%     (\#)
        >{\raggedright\arraybackslash}p{3.0cm}%   (Component)
        >{\raggedright\arraybackslash}p{3.0cm}%   (Potential Failure)
        >{\raggedright\arraybackslash}p{3.0cm}%   (Effect on User)
        >{\raggedright\arraybackslash}p{3.0cm}%   (Likely Cause)
        >{\centering\arraybackslash}p{1.5cm}%   (Severity)
        >{\centering\arraybackslash}p{2.4cm}%     (Occurence Frequency)
        >{\raggedright\arraybackslash}p{3.0cm}%     (Detection Method)
        >{\centering\arraybackslash}p{2.0cm}%     (Detection Likelihood)
        >{\centering\arraybackslash}p{2.4cm}%   (Priority)
        >{\raggedright\arraybackslash}p{3.0cm}%     (Recommended Mitigation)
        >{\raggedright\arraybackslash}p{3.0cm}%
      }

    \toprule
    \textbf{\#} & 
    \textbf{Component / Function} & 
    \textbf{Potential Failure Mode} &
    \textbf{Effect on User / System}&
    \textbf{Likely Cause(s)} & 
    \textbf{Severity} &
    \textbf{Occurrence Frequency} &
    \textbf{Detection Method} &
    \textbf{Detection Likelihood} &
    \textbf{Priority \hspace{15pt}(S + O + D)} &
    \textbf{Recommended Action} &
    \textbf{Relevant Requirement(s)}\\
    \midrule
    \endfirsthead
    \toprule
    
    \midrule
    \endhead


    1 & Microphone &  Unresponsive / Distorted Microphone. & Not able
    to effectively localize audio. \cellrule Unable to provide warnings to user. & 
    Microphone circuit failure. \cellrule Microphone damage. \cellrule Excessive ambient noise. &
    5 & 1  &
    Multiple corrupted microphone data frames. \cellrule Excessive white noise detection.
    \cellrule Short circuit detection for microphones. & 3 & 9 & Detect failure
    using microphone audio (or lack thereof) and notify user. & FR1.4, FR2.3, FR3.5, FR7.2, NFR2.1  \\

    \midrule
    2 & Visualization Controller & Disconnection from display. & Unable to provide 
    visual notifications to user. & Depending on connection type: cable, 
    wireless interference, dropped connection. & 3 & 2 &
    Loss of connection signal. \cellrule Failure to send data to display. & 1 & 6 &
    There are no safety requirements that can mitigate this hazard. & NFR2.1\\

    \midrule


    3 & Embedded Firmware & System crashes and freezes. & The system remains unusable
    until it is restarted. \cellrule Unable to provide notifications to user. &
    Software bugs. \cellrule Insufficient Error Handling \cellrule Insufficient
    Requirements. & 5 & 3 & System watchdog timer 
    \cellrule Halt Interrupt. & 1 & 9 & Implement a watchdog timer to reset the
    system in the event of a crash. \cellrule Implement adequate logging to
    diagnose the cause of firmware failure during post-mortem. \cellrule Robust testing and 
    error handling. & FR1.3, FR4.1, FR4.4 \\
    \midrule

    4 & Sound Detection (Audio360 Engine)& Failure to detect important sounds. & Failure to notify
    user of critical sounds. & Insufficient microphone quality. \cellrule Poor 
    classification algorithm. \cellrule Excessive ambient noise. & 4 & 3 &
    This is difficult to detect without extensive testing. \cellrule Impossible
    to detect in production. & 5 & 12 & Have a thorough
    library of sounds to detect and test against. \cellrule Measure microphone 
    performance on sound library. & FR3.5, FR4.4, FR5.1, FR6.1  \\

    \midrule

    5 & Sound Classifier (Audio360 Engine) & Misclassification of sounds. & Notify the user of the 
    wrong sounds. & Insufficient classification algorithm \cellrule Insufficient
    microphone quality. & 3 & 2 & This is difficult to detect
    without extensive user testing.\cellrule Impossible to detect in production. & 5 & 10 & Extensive testing in real-world
    environments and simulation. \cellrule Verify microphone quality during operation.
    & FR4.4, FR5.1, FR5.4, NFR5.1, NFR5.2 \\


    \midrule

    6 & Sound Localizer (Audio360 Engine) & Inaccurate direction determination for sounds. & Misinform
    the user of the direction of important sounds. & Incorrect localization algorithm.
    \cellrule Insufficient microphone quality. & 4 & 2 & This is
    difficult to detect without extensive localization testing. \cellrule Impossible to detect
    in production. & 5 & 11 & Extensive localization testing in real-world environments and simulation.
    & FR5.2, NFR5.3 \\
    
    \bottomrule
    \end{longtable}

\restoregeometry

\pdfpagewidth=8.5in
\pdfpageheight=11in

\subsection{Out of Scope Failure Modes}
The project involves off the shelf hardware components with their own possible
failure modes and possible mitigations. Since the hardware and electrical 
components are not being designed as part of the project, 
these failure modes are considered out of scope:
\begin{itemize}
    \item Battery failure / damage.
    \item Physical display damage.
    \item Power supply issues.
    \item Physical damage to the glasses.
\end{itemize}


\section{Safety and Security Requirements}

\begin{itemize}
    \item \label{SR1}\textbf{SR1:} The microcontroller shall run the main
    \progname ~software in a closed environment. Only the approved
    microphones and output display shall interface to it. This ensures
    protection against unauthorized access.

    \item \label{SR2}\textbf{SR2:} The system shall permanently discard
    microphone audio data immediately after completion of audio analysis.
    This requirement ensures data privacy and prevents unauthorized reuse of
    raw audio.

    \item \label{SR3}\textbf{SR3:} The system shall
    detect and flag audio anomalies such as clipping, lost signal and silence.
    This enables identification of microphone or signal path faults.

    \item \label{SR4}\textbf{SR4:} The system shall disable audio classification
    and directional analysis features until microphone faults addressed in
    \hyperref[SR3]{SR3} are resolved. This prevents the generation of
    unreliable or unsafe outputs.

    \item \label{SR5}\textbf{SR5:} The visualization controller 
    shall present information in a non-intrusive manner, minimizing visual
    obstruction so users can safely perform external activities.

    \item \label{SR6}\textbf{SR6:} The visualization controller component
    shall alert users when critical features such as sound classification or
    direction determination fail. This ensures that users are aware of degraded
    safety functions.

    \item \label{SR7}\textbf{SR7:} The system shall have
    an audio processing success rate end to end of at least 90\% over 60 seconds.
    This mitigates classification related hazards identified in the FMEA.

    \item \label{SR8}\textbf{SR8:} The system shall notify users when a sound 
    classification result has low confidence or is unrecognized to prevent 
    misleading contextual feedback.

    \item \label{SR9}\textbf{SR9:} The system shall estimate the
    direction of arrival of an audio source with a maximum error of 45 degrees.
    This ensures reliable directional awareness for the user.

    \item \label{SR10}\textbf{SR10:} The system shall perform continuous
    diagnostics on all hardware components to monitor hardware errors in
    real-time. This ensures the system can react to failures as soon as they
    occur.
    
\end{itemize}

\section{Roadmap}

The following safety and security requirements are planned for implementation 
within the scope of the capstone project timeline:
\begin{itemize}
    \item \hyperref[SR2]{SR2}
    \item \hyperref[SR3]{SR3}
    \item \hyperref[SR5]{SR5}
    \item \hyperref[SR7]{SR7}
    \item \hyperref[SR9]{SR9}
\end{itemize}

The following safety and security requirements are identified as stretch goals. 
Their implementation will depend on the available development bandwidth and 
project progress:
\begin{itemize}
    \item \hyperref[SR4]{SR4}
    \item \hyperref[SR6]{SR6}
    \item \hyperref[SR8]{SR8}
    \item \hyperref[SR10]{SR10}
\end{itemize}

\newpage
\bibliographystyle{IEEEtran}
\bibliography{../../refs/References}

\newpage{}

\section*{Appendix --- Reflection}

\begin{enumerate}
    \item What went well while writing this deliverable? 
    
    \textbf{Sathurshan:} The team had a strong understanding of the
    safety-critical nature of the system. Some members of the team also had
    experience in writing a Hazard Analysis document from extra-curricular
    activities. This made performing the hazard analysis more straightforward,
    as we were able to effectively identify and evaluate potential risks within
    the system.

    \textbf{Kalp:} I think having written the SRS document before this one made
    the hazard analysis more straightforward, as we had a clear understanding of
    the system and the risks associated with it. We had many discussions during
    the SRS document that really clarified the requirements and the expected 
    software states of the system, really helping us quickly go through this 
    document without much difficulty.
    
    \textbf{Nirmal Chaudhari:} For this deliverable, I focused on the critical
    assumptions section. What worked well while writing this deliverable was 
    having a clear enough picture of constraints that exist with this project 
    from the environment section of the SRS doc. From these constraints, it 
    became clear what is out of our hands, and needs assumptions for our project 
    to validate the requirements we have set. 

    
    \textbf{Omar:} I think what went well when writing this document was the 
    cross collaboration between myself and Sathurshan. We were able to have
    in depth discussions about the potential hazards and risks from different 
    components since he was responsible for most of the technical requirements.
    This was instrumental in ensuring a detailed and accurate FMEA table.

    \textbf{Jay:} The FMEA structure provided a clear framework for thinking 
    through potential failures systematically. Breaking down each component and 
    considering what could go wrong helped us identify risks we might have 
    otherwise missed. The team's collaborative approach also meant we could 
    challenge each other's assumptions and ensure we weren't overlooking 
    critical failure modes.

    \item What pain points did you experience during this deliverable, and how
    did you resolve them?

    \textbf{Sathurshan:} Some sections of this deliverable were dependent on the
    completion of other sections within this document and the SRS. As a result,
    managing the timeline for these sections was challenging. To resolve this,
    we coordinated closely with the owners of the dependent sections, which
    improved collaboration and allowed us to exchange constructive feedback to
    ensure consistency across the document.

    \textbf{Kalp:} I think the main pain point was that there was high 
    dependence between the sections of the document. This lead to us assigning
    the tasks to each person in large chunks (to avoid the issues we ran into 
    during the SRS document), but then that left us with people not having much
    to contribute to on the document. We ended up having to reshuffle the tasks
    around to help mitigate this issue. 
    
    \textbf{Nirmal Chaudhari:} While working on this deliverable, it sometimes 
    became unclear what should be listed as an assumption vs constraint. As we 
    continued brainstorming however, we started to look at it from another angle 
    where the assumptions are technically just listed as boundaries that our 
    system has, and requires assumptions. Moreover, referencing between the 
    sections other people are working on for consistency was a difficult task 
    to do, especially within a small time frame.  

    \textbf{Omar:} One pain point I experienced was ensuring that the FMEA table
    was well formatted and readable. Adjusting the latex code to ensure that
    the table fit within the page margins while still being legible was a
    challenge. I resolved this by experimenting with different table types
    and packages, even adjusting page sizes till it worked. 

    \textbf{Jay:} Initially, it was difficult to assess the severity and 
    likelihood of different failure modes objectively. We had to balance 
    being thorough without being overly pessimistic about every possible 
    edge case. I resolved this by focusing on realistic operating conditions 
    and consulting with teammates who have more hardware experience to ground 
    my estimates in practical constraints.

    \item Which of your listed risks had your team thought of before this
    deliverable, and which did you think of while doing this deliverable? For
    the latter ones (ones you thought of while doing the Hazard Analysis), how
    did they come about?

    \textbf{Sathurshan:} Before this deliverable, I had already identified the
    risk of microphone failures, as it serves as the system's primary input.
    During the hazard analysis, we also identified the risk of unauthorized
    access to system components, particularly data stored on the microcontroller
    and modification of the software. This risk emerged as we considered
    non-physical hazards, which the following questions allude to.

    \textbf{Kalp:} A big risk that I think the team was was already considering
    before the deliverable was risk of hardware failure (microphone, 
    smart glasses, microcontroller, etc.). This was mainly due to us just 
    thinking mainly about the high level system, with the hardware being major
    components of that mental model. During the document is when I would say we
    started to think more about security and privacy risks, specifically with 
    data access breach and unauthorized access to the system.
    
    \textbf{Nirmal Chaudhari:} One of the listed risks we thought of before 
    the deliverable was not being able to find a microcontroller with 4 ADCs 
    with the limited budget we had. This risk was highlighted to us by our 
    supervisor from our initial meeting. One risk we though about while doing 
    this deliverable was risks associated with the environment, which we have 
    no control over. During our team brainstorming section, we realized that 
    weather conditions like rain, humidity, wind are things our system will 
    have to be robust enough to handle given the limited budget we have. 
    To address these risks, we added in our critical assumption that for the 
    scope of this project we will have perfect weather conditions. 

    \textbf{Omar:} Prior to this deliverable, I had already considered the risk
    of system crashes and freezes due to my experience with embedded systems.
    During the hazard analysis, we had to delve further into the specific 
    connection mechanism that the smart glasses provide. This led to the 
    identification of the risk of disconnection from the display, which I had
    not previously considered.

    \textbf{Jay:} We had already considered the risk of classification errors 
    and inaccurate direction estimation since those are core to the system's 
    value proposition. What emerged during the hazard analysis was thinking 
    about failures, where one component's malfunction could propagate 
    through the system and cause misleading outputs rather than obvious failures. 
    This came from systematically working through the FMEA table and considering 
    how each failure mode affects downstream components.

    \item Other than the risk of physical harm (some projects may not have any
    appreciable risks of this form), list at least 2 other types of risk in
    software products. Why are they important to consider?

    \textbf{Sathurshan:} Other forms of software-related risks include data
    privacy violations and security vulnerabilities that allow unauthorized
    access without user consent. These are critical to consider because they can
    user data can be used against them, resulting in ethical consequences.

    \textbf{Kalp:} I think two forms of software-related risks that are 
    important to consider are data privacy violations and reliability issues.
    With many software systems dealing with data from the user, or providing
    important data to the user, it's important to consider the risks of data 
    being leaked / breached (make private information public or inform the 
    user something wrong potentially misguiding their actions).

    \textbf{Nirmal Chaudhari:} One other type of risk that exists with software 
    projects is security and privacy risks. While working on the environment 
    section of the SRS document, I realized there are a lot of legal constraints
     around collecting and analyzing sounds in a given environment. This is 
     important to consider, since if its left unaddressed moving the system into
      a production environment would be impossible. Another type of risk that 
      exists is environmental risks. If the battery we choose is inefficient or 
      is harmful to the environment when being disposed of, this would 
      negatively impact the environment around us. 

    \textbf{Omar:} Two other types of software-related risks include
    performance degradation over time. This is an important issues with
    emerging technologies that rely on machine learning models, as they may
    become less effective as the environment and sensors change overtime, even
    on the same hardware. Another risk is bad UI/UX design, which can lead to 
    user frustration and abandonment of the product. This is especially
    important to consider for assistive technologies, as they need to be
    intuitive and easy to use for individuals with varying levels of ability.

    \textbf{Jay:} Two important risks are loss of user confidence and poor code 
    maintainability. If the system gives false alarms or inconsistent results, 
    users will stop trusting it and won't rely on it when they actually need it. 
    Maintainability is also critical because if the code is messy or poorly 
    organized, fixing bugs later becomes risky and time-consuming, which is 
    especially problematic for a safety device that needs to stay reliable 
    over time.
\end{enumerate}

\end{document}