\documentclass{article}

\usepackage{tabularx}
\usepackage{booktabs}

\title{Problem Statement and Goals\\\progname}

\author{\authname}

\date{}

%% Comments

\usepackage{color}

\newif\ifcomments\commentstrue %displays comments
%\newif\ifcomments\commentsfalse %so that comments do not display

\ifcomments
\newcommand{\authornote}[3]{\textcolor{#1}{[#3 ---#2]}}
\newcommand{\todo}[1]{\textcolor{red}{[TODO: #1]}}
\else
\newcommand{\authornote}[3]{}
\newcommand{\todo}[1]{}
\fi

\newcommand{\wss}[1]{\authornote{magenta}{SS}{#1}} 
\newcommand{\plt}[1]{\authornote{cyan}{TPLT}{#1}} %For explanation of the template
\newcommand{\an}[1]{\authornote{cyan}{Author}{#1}}

%% Common Parts

\newcommand{\progname}{Software Engineering} 
\newcommand{\authname}{Team \#6, Six Sense
\\ Omar Alam
\\ Sathurshan Arulmohan
\\ Nirmal Chaudhari
\\ Kalp Shah
\\ Jay Sharma
}        

\usepackage{hyperref}
    \hypersetup{colorlinks=true, linkcolor=blue, citecolor=blue, filecolor=blue,
                urlcolor=blue, unicode=false}
    \urlstyle{same}
                                


\begin{document}

\maketitle

\begin{table}[hp]
\caption{Revision History} \label{TblRevisionHistory}
\begin{tabularx}{\textwidth}{llX}
\toprule
\textbf{Date} & \textbf{Developer(s)} & \textbf{Change}\\
\midrule
Date1 & Name(s) & Description of changes\\
Date2 & Name(s) & Description of changes\\
... & ... & ...\\
\bottomrule
\end{tabularx}
\end{table}

\section{Problem Statement}

\wss{You should check your problem statement with the
\href{https://github.com/smiths/capTemplate/blob/main/docs/Checklists/ProbState-Checklist.pdf}
{problem statement checklist}.} 

\wss{You can change the section headings, as long as you include the required
information.}

\subsection{Problem}

\subsection{Inputs and Outputs}

\wss{Characterize the problem in terms of ``high level'' inputs and outputs.  
Use abstraction so that you can avoid details.}

\subsection{Stakeholders}

\subsection{Environment}

\wss{Hardware and Software Environment}

\section{Goals}

\section{Stretch Goals}

\section{Extras}

% TODO: link to the actual documentation when we create them.
\begin{enumerate}
    \item Price + Hardware Selection Report
    \item Usability Report
\end{enumerate} 

\newpage{}

\section*{Appendix --- Reflection}

\wss{Not required for CAS 741}

The purpose of reflection questions is to give you a chance to assess your own
learning and that of your group as a whole, and to find ways to improve in the
future. Reflection is an important part of the learning process.  Reflection is
also an essential component of a successful software development process.  

Reflections are most interesting and useful when they're honest, even if the
stories they tell are imperfect. You will be marked based on your depth of
thought and analysis, and not based on the content of the reflections
themselves. Thus, for full marks we encourage you to answer openly and honestly
and to avoid simply writing ``what you think the evaluator wants to hear.''

Please answer the following questions.  Some questions can be answered on the
team level, but where appropriate, each team member should write their own
response:


\begin{enumerate}
    \item What went well while writing this deliverable? 
    
    \textbf{Omar Alam:} I think all members of our team were proactive and genuinely interested in the project presented which made it easier
    to delegate and expect high quality work. 
    
    \textbf{Jay Sharma:} We aligned quickly on the problem and split responsibilities in a way that played to each person's strengths. 
    Our LaTeX/Git workflow came together smoothly, which helped us iterate fast on the documents.
    \item What pain points did you experience during this deliverable, and how
    did you resolve them?
    \textbf{Omar Alam:} Since the project idea incorporates glasses with displays that are visible to the user, we had to do a significant amount
    of research to figure out if it was feasible in the time that we have. We resolved this by developing a contingency plan that would allow us to 
    still allow us to develop the core algorithms without the display glasses.

    \textbf{Jay Sharma:} Our main pain point was deciding on the hardware that we would use for the project. We had to spend a significant amount of time
     researching the different options and their feasibility.
    \item How did you and your team adjust the scope of your goals to ensure
    they are suitable for a Capstone project (not overly ambitious but also of
    appropriate complexity for a senior design project)?
    \textbf{Omar Alam:} My team and I spent a significant amount of time researching the feasibility of the project. Since the team does not have much
    experience with signal processing, we decided to consult with Dr. Mohrenschildt to get his opinion on the project. He provided us with valuable feedback on how
    to constraint our project goals to ensure that we can complete the project in the time we have.
    
    \textbf{Jay Sharma:} We started by mapping the problem space, did feasibility analyses, and set criteria around impact, difficulty, and time. 
    We also checked in with Prof. Mohrenschildt to advise the direction of our project and keep the scope focused for the capstone timeline.
\end{enumerate}  

\end{document}