% THIS DOCUMENT IS TAILORED TO REQUIREMENTS FOR SCIENTIFIC COMPUTING.  IT SHOULDN'T
% BE USED FOR NON-SCIENTIFIC COMPUTING PROJECTS
\documentclass[12pt]{article}

\usepackage{amsmath, mathtools}
\usepackage{amsfonts}
\usepackage{amssymb}
\usepackage{graphicx}
\usepackage{colortbl}
\usepackage{xr}
\usepackage{hyperref}
\usepackage{longtable}
\usepackage{xfrac}
\usepackage{tabularx}
\usepackage{float}
\usepackage{siunitx}
\usepackage{booktabs}
\usepackage{caption}
\usepackage{pdflscape}
\usepackage{afterpage}

\usepackage[round]{natbib}

%\usepackage{refcheck}

\hypersetup{
    bookmarks=true,         % show bookmarks bar?
      colorlinks=true,       % false: boxed links; true: colored links
    linkcolor=red,          % color of internal links (change box color with linkbordercolor)
    citecolor=green,        % color of links to bibliography
    filecolor=magenta,      % color of file links
    urlcolor=cyan           % color of external links
}

%% Comments

\usepackage{color}

\newif\ifcomments\commentstrue %displays comments
%\newif\ifcomments\commentsfalse %so that comments do not display

\ifcomments
\newcommand{\authornote}[3]{\textcolor{#1}{[#3 ---#2]}}
\newcommand{\todo}[1]{\textcolor{red}{[TODO: #1]}}
\else
\newcommand{\authornote}[3]{}
\newcommand{\todo}[1]{}
\fi

\newcommand{\wss}[1]{\authornote{magenta}{SS}{#1}} 
\newcommand{\plt}[1]{\authornote{cyan}{TPLT}{#1}} %For explanation of the template
\newcommand{\an}[1]{\authornote{cyan}{Author}{#1}}

%% Common Parts

\newcommand{\progname}{Software Engineering} 
\newcommand{\authname}{Team \#6, Six Sense
\\ Omar Alam
\\ Sathurshan Arulmohan
\\ Nirmal Chaudhari
\\ Kalp Shah
\\ Jay Sharma
}        

\usepackage{hyperref}
    \hypersetup{colorlinks=true, linkcolor=blue, citecolor=blue, filecolor=blue,
                urlcolor=blue, unicode=false}
    \urlstyle{same}
                                


% For easy change of table widths
\newcommand{\colZwidth}{1.0\textwidth}
\newcommand{\colAwidth}{0.13\textwidth}
\newcommand{\colBwidth}{0.82\textwidth}
\newcommand{\colCwidth}{0.1\textwidth}
\newcommand{\colDwidth}{0.05\textwidth}
\newcommand{\colEwidth}{0.8\textwidth}
\newcommand{\colFwidth}{0.17\textwidth}
\newcommand{\colGwidth}{0.5\textwidth}
\newcommand{\colHwidth}{0.28\textwidth}

% Numbering 
\usepackage{amsthm}
\usepackage{xassoccnt}
%\newtheorem{req}{Requirement}[section]        
\newtheorem{req}{Requirement}        
\theoremstyle{definition}        
\newtheorem{constraint}{Constraint}
\newtheorem{goal}{Goal}
\DeclareCoupledCountersGroup{theorems}
\DeclareCoupledCounters[name=theorems]{req,constraint,goal}
\setcounter{goal}{0}

\usepackage{fullpage}


\begin{document}

\title{Software Requirements Specification for \progname: subtitle describing software} 
\author{\authname}
\date{\today}
	
\maketitle

~\newpage

\pagenumbering{roman}

\tableofcontents

~\newpage

\section*{Revision History}


\begin{tabularx}{\textwidth}{p{3cm}p{2cm}X}
\toprule {\bf Date} & {\bf Version} & {\bf Notes}\\
\midrule
2025-10-06 & 1.0 & Initial Write-up\\
\bottomrule
\end{tabularx}


~\newpage

\section{Goal}

\subsection{G.1 Context and overall objective}
\wss{High-level view of the project: organizational context and reason for building a system.} 

\begin{goal}\label{goal:first}
This is a goal example. If you need explicit (and automatic) numbering, you can use the definitions in the \texttt{.tex} template.
\end{goal}

\begin{req}\label{req:cross}
This is a requirement example. It illustrates how numbering is continuous and cross-types (if this is what you need).
\end{req}
    
\subsection{G.2 Current situation}
\wss{Current state of processes to be addressed by the project and the resulting system.}

\begin{req}\label{req:memo}
This is a requirement example. 
It refines \ref{goal:first}
\end{req}
    
\subsection{G.3 Expected benefits}
\wss{New processes, or improvement to existing processes, made possible by the project’s results.}


\subsection{G.4 Functionality overview}
\wss{Overview of the functions (behavior) of the system. Principal properties only (details are in the System book).}

\subsection{G.5 High-level usage scenarios}
\wss{Fundamental usage paths through the system.}

\subsection{G.6 Limitations and exclusions}
\wss{Aspects that the system need not address.}

\subsection{G.7 Stakeholders and requirements sources}
\wss{Groups of people who can affect the project or be affected by it, and other places to consider for information about the project and system.}


\section{Environment}

\subsection{E.1 Glossary}
\wss{Clear and precise definitions of all the vocabulary specific to the application domain, including technical terms, words from ordinary language used in a special meaning, and acronyms.
This chapter should not be empty!}

\subsection{E.2 Components}
\wss{List of elements of the environment that may affect or be affected by the system and project. Includes other systems to which the system must be interfaced.}

\subsection{E.3 Constraints}
\wss{Obligations and limits imposed on the project and system by the environment.}

\subsection{E.4 Assumptions}
\wss{Properties of the environment that may be assumed, with the goal of facilitating the project and simplifying the system.}

\subsection{E.5 Effects}
\wss{Elements and properties of the environment that the system will affect.}

\subsection{E.6 Invariants}
\wss{Properties of the environment that the system’s operation must preserve.}

\section{System}

\subsection{S.1 Components}
\wss{Overall structure expressed by the list of major software and, if applicable, hardware parts.}

\subsection{S.2 Functionality}
\wss{One section, S.2.n, for each of the components identified in S.2, describing the corresponding behaviors (functional and non-functional properties).}

\subsection{S.3 Interfaces}
\wss{How the system makes the functionality of S.2 available to the rest of the world, particularly user interfaces and program interfaces (APIs).}

\subsection{S.4 Detailed usage scenarios}
\wss{Examples of interaction between the environment (or human users) and the system: use cases, user stories.}

\subsection{S.5 Prioritization}
\wss{Classification of the behaviors, interfaces and scenarios (S.2, S.3 and S.4) by their degree of criticality.}

\subsection{S.6 Verification and acceptance criteria}
\wss{Specification of the conditions under which an implementation will be deemed satisfactory.}

\section{Project}

\subsection{P.1 Roles and personnel}
\wss{Main responsibilities in the project; required project staff and their needed qualifications.}

\subsection{P.2 Imposed technical choices}
\wss{Any a priori choices binding the project to specific tools, hardware, languages or other technical parameters.}

\subsection{P.3 Schedule and milestones}
\wss{List of tasks to be carried out and their scheduling.}

\subsection{P.4 Tasks and deliverables}
\wss{Details of individual tasks listed under P.3 and their expected outcomes.}

\subsection{P.5 Required technology elements}
\wss{External systems, hardware and software, expected to be necessary for building the system.}

\subsection{P.6 Risks and mitigation analysis}
\wss{Potential obstacles to meeting the schedule of P.4, and measures for adapting the plan if they do arise.}

\subsection{P.7 Requirements process and report}
\wss{Initially, description of what the requirements process will be; later, report on its steps.}


\newpage{}
\section*{Appendix --- Reflection}

The purpose of reflection questions is to give you a chance to assess your own
learning and that of your group as a whole, and to find ways to improve in the
future. Reflection is an important part of the learning process.  Reflection is
also an essential component of a successful software development process.  

Reflections are most interesting and useful when they're honest, even if the
stories they tell are imperfect. You will be marked based on your depth of
thought and analysis, and not based on the content of the reflections
themselves. Thus, for full marks we encourage you to answer openly and honestly
and to avoid simply writing ``what you think the evaluator wants to hear.''

Please answer the following questions.  Some questions can be answered on the
team level, but where appropriate, each team member should write their own
response:


Questions 3 and 5 are answered as a team.

\begin{enumerate}
  \item What went well while writing this deliverable? 

  \textbf{Kalp:} I think what went really well during the writing of the SRS
  document was the frequent in-person work sessions. We were all able to sit
  together in a room, quickly review each other's work, make changes, discuss
  unclear assignments, and so on. This process, compared to the individual
  writeup and review process we used earlier, was much more efficient and
  effective, especially since this document was very interdependet (goals
  section for example having content related to the requirements section).
  
  \textbf{Nirmal Chaudhari:} Coming up with the schedule and milestone for this
  project was relatively easy for the elicitation and documentation stages since
  it was basically just a recap and coherent with what we just went through.
  Moreover, since our team had come up with the list of focus areas in advance,
  coming up with team roles and added responsibilities was relatively easy as
  well. 
  
  \textbf{Sathurshan:} In this deliverable, the team effectively collaborated on
  specific sections, allowing us to provide constructive feedback and
  iteratively refine our requirements during the initial write-up. This process
  helped align our goals for the system and provided a clearer understanding of
  what needed to be achieved.

  \textbf{Omar}: Throughout the deliverable writing process, the team
  communicated effectively and responded to feedback on the pull requests in a
  timely manner. This allowed us to rapidly iterate on the document. Since
  everyone brings an equal level of commitment and enthusiasm to the project, it
  made the collaboration process smooth and enjoyable.

  \textbf{Jay:} Building on our earlier deliverables made the writing process 
  much smoother. Having already defined the problem scope and development plan 
  gave us a solid foundation to work from, so translating those ideas into 
  detailed requirements felt natural. The structured template also helped keep 
  everything organized, making it easier to ensure completeness across all 
  sections. 

  \item What pain points did you experience during this deliverable, and how did
  you resolve them?

  \textbf{Kalp:} I think the main pain point was the way that we divided up the
  work on the document since many of the sections were dependent on each other.
  This often lead to some people on the team waiting for others to finish their
  section before they could start their own so that there wasn't conflicting
  information or text in the document. This process was slighly improved with
  the frequent in-person work sessions, but it was still a pain point.
  
  \textbf{Nirmal Chaudhari:} While working on the environment section, intially
  coming up with the list of components in the environment the system will have
  to interact with as difficult. This is because of the ambuguity that initially
  existed with what we can consider the "environment" for this system. This
  unclearity was resolved by coming up with a very high level use-case scenario
  of what a typical person would be interacting with when using the device.
  \textbf{Sathurshan:} Many sections were dependent on others being completed
  first, which blocked some of the writing process. With the granted extension,
  several sections were delayed, leading to a time crunch toward the end. To
  address this, the team organized multiple collaborative work sessions to work
  on the SRS document together. This allowed us to exchange ideas in real time,
  resolve blockers quickly, and progress in parallel. Moving forward, during the
  project planning stage, the team should prioritize dependency related issues
  and set internal deadlines to ensure smoother progress.

  \textbf{Omar}: There was some friction when it came to sections that relied on
  other sections being completed first. However, the team was able to work
  through these issues by holding several in-person and virtual meetings to
  discuss the document and make progress together. This allowed us to quickly
  resolve any blockers and ensure that everyone was on the same page.

  \textbf{Jay:} Striking the right balance between technical precision and 
  readability was challenging. Some sections needed enough detail to be 
  actionable, but too much made them hard to follow. We addressed this by 
  reviewing each other's sections and providing feedback on clarity, which 
  helped us converge on a consistent level of detail throughout the document. 

  \item How many of your requirements were inspired by speaking to your
  client(s) or their proxies (e.g. your peers, stakeholders, potential users)?

  A lot of the requirements related to focus areas defined in our SRS document
  were inspired by our project supervisor. He gave great insight into what major
  components need to be researched for this system to work. For example, he went
  over the requirement of needing 4 ADC converters in our microprocessor to
  retrieve synchronized audio input across all 4 microphones. If he didn't give
  this insight early on, the team would have been stuck in the later stages on
  the project with a microprocessor that will not work well for this system.
  Furthermore, he gave good information of how the team can go about using
  Independent Component Analysis to seperate audio into sources. He also
  mentioned we shouldn't use deep machine learning models for audio
  classification, since they won't be able to run on the microprocessor well. 

  
  \item Which of the courses you have taken, or are currently taking, will help
  your team to be successful with your capstone project.

  \textbf{Nirmal Chaudhari: } For this project the three courses I think will
  enable us to be the most successful are: Signals \& Systems, Requirements
  Engineering and Software Design 2. Signals \& Systems was important since the
  entire project revolves around processing and analyzing audio signals in the
  frequency domain. Requirements Engineering is important in helping us figure
  out our requirements and ensuring throughout the entire process that what we
  are building is the right thing. And Software Design 2 is useful in helping us
  implement best practices into the project, thus making it sustainable in the
  future.

  \textbf{Sathurshan:} 3MX3: signal processing, 2GA3: computer hardware
  architecture. 3RA3: software requirements, 2DA4: software design, 3A04:
  software architecture, 3S03: software testing.

  \textbf{Omar}: The courses that will help us the most with this project are:
  Signals and Systems (3MX3) - This course provides a solid foundation in signal
  processing, which is crucial for our project focused on audio signal analysis.
  Concurrent Programming (3BB4) - This course teaches us the core principles of
  concurrent programming which will be essential for implementing real-time
  processing on our microcontroller.

  \textbf{Kalp:} The courses that will probably help us the most for this
  project are 3MX3 (Signals \& Systems with Dr. Mohrenschildt). This course
  taught us many of the signal processing algorithms for that we will likely
  need to apply during our audio analysis for the system. Another good course
  would be 3A04 (Software Architecture) that taught us how to design and
  implement large scale software systems which will be important for our project
  as we we will be designing many modular components that work together. The
  planning techniques from the course, specifically, will be very useful.

  \textbf{Jay:} 3MX3 (Signals \& Systems) is definitely critical for 
  understanding audio processing and frequency analysis. Beyond that, 
  3BB4 (Concurrent Systems) will be valuable for managing real-time 
  constraints on the microcontroller, and 2GA3 (Computer Architecture) 
  will help us optimize performance on embedded hardware. The software 
  engineering courses like 3RA3 and 3S03 are also important for ensuring 
  we build a reliable, well-tested system.

  \item What knowledge and skills will the team collectively need to acquire to
  successfully complete this capstone project?  Examples of possible knowledge
  to acquire include domain specific knowledge from the domain of your
  application, or software engineering knowledge, mechatronics knowledge or
  computer science knowledge.  Skills may be related to technology, or writing,
  or presentation, or team management, etc.  You should look to identify at
  least one item for each team member.

  Each memmber of the team requires the following qualfiications as contributing
  developers to the team.

  \begin{itemize}
    \item Embedded software development.
    \item Strong software design for features being implemented on any
    microcontroller platform. 
    \item Strong testing skills.
    \item Strong debugging skills. 
  \end{itemize}

  In addition to this, since each team member is a focus area expert, they
  require the following skills and competencies to carry out that role.

  \begin{itemize}\setlength\itemsep{4pt}\setlength{\leftmargini}{2em}
    \item Look through research articles, and technical evaluations to come up
    with feasibile approaches for proposed methods. 
    \item Collaborate with other team members to discuss findings. 
    \item Maintain clear and organized documentation of sources and proposed
    methods. 
    \item Ensure that all research and implementation choices align with project
     objectives, timeline and budgeting costs. 
    \item Based on the confirmed approach, complete the full implementation of
    that focus in the system. 
    \item After implementation, create test cases that cover's the main
    functionality of the feature in that focus area. 
    \item Configure github pipeline to run those tests on every PR and merge
    into a feature branch. 
  \end{itemize}


  \item For each of the knowledge areas and skills identified in the previous
  question, what are at least two approaches to acquiring the knowledge or
  mastering the skill?  Of the identified approaches, which will each team
  member pursue, and why did they make this choice?

  \textbf{Kalp:} I think the main focus for knowledge areas to explore has to be
  around embedded software development and hardware integration. Since I've only
  done software development in industry before, I have experience developing, 
  testing, and debugging software, but have never explored the hardware side of
  the systems. 

  \textbf{Nirmal Chaudhari:} For embedded software development and strong
  software design, two approaches are available: (1) reviewing processor
  documentation and (2) practicing and test building small programs on the board
  to see how it targets key peripherals work. I will focus on the second
  approach since practical experience seems more important. For testing and
  debugging skills, the team can (1) study or use existing knowledge of
  frameworks and (2), conduct peer reviews with other members on the team. For
  this I would prefer the second approach since working with the team on real
  bugs would help me grow as a developer to see how others resolve bugs. For
  research and technical evaluation, the knowledge can be gained by (1) reading
  research papers existing in the focus area and (2) contacting domain experts
  like mvm for insights. I plan to start with the first option, since academic
  resources provides structure that can be referenced to later on.
 
  \textbf{Sathurshan:} I plan to focus on gaining knowledge in embedded software
  development, as it is something I want to specialize in the next few years.

  \textbf{Omar}: I believe the best way to learn any skill is by doing it and
  struggling through problems. Each microcontroller platform has its own quirks
  and tools. I plan to gain practical experience by developing smaller projects
  on the STM32 platform, which will help me understand its architecture. Each
  problem in our project can be subdivided into smaller projects, which will
  help me learn as I go. Additionally, I will refer to the STM32 documentation
  and online tutorials to supplement my learning.

  \textbf{Jay:} I'll be focusing on Independent Component Analysis (ICA) for 
  audio source separation. Two main approaches are: (1) implementing ICA 
  algorithms from research papers and testing them with sample audio data, 
  and (2) prototyping directly on the hardware with real microphone input. 
  I'll start with the first approach since it lets me validate the core 
  algorithms quickly without hardware dependencies, then transition to 
  hardware testing once we have a stable foundation. I'll also lean on our 
  supervisor's expertise to guide algorithm selection and avoid overly 
  complex solutions.

\end{enumerate}


\end{document}