Questions 3 and 5 are answered as a team.

\begin{enumerate}
  \item What went well while writing this deliverable? 

  \textbf{Kalp:} I think what went really well during the writing of the SRS
  document was the frequent in-person work sessions. We were all able to sit 
  together in a room, quickly review each other's work, make changes, discuss 
  unclear assignments, and so on. This process, compared to the individual 
  writeup and review process we used earlier, was much more efficient and 
  effective, especially since this document was very interdependet (goals
  section for example having content related to the requirements section).
  
  \textbf{Nirmal Chaudhari:} Coming up with the schedule and milestone for this
  project was relatively easy for the elicitation and documentation stages 
  since it was basically just a recap and coherent with what we just went 
  through. Moreover, since our team had come up with the list of focus areas 
  in advance, coming up with team roles and added responsibilities was 
  relatively easy as well. 
  
  \textbf{Sathurshan:} In this deliverable, the team effectively collaborated on
  specific sections, allowing us to provide constructive feedback and
  iteratively refine our requirements during the initial write-up. This process
  helped align our goals for the system and provided a clearer understanding of
  what needed to be achieved.

  \textbf{Omar}: Throughout the deliverable writing process, the team communicated
  effectively and responded to feedback on the pull requests in a timely manner. 
  This allowed us to rapidly iterate on the document. Since everyone brings 
  an equal level of commitment and enthusiasm to the project, it made the 
  collaboration process smooth and enjoyable.

  \item What pain points did you experience during this deliverable, and how did
  you resolve them?

  \textbf{Kalp:} I think the main pain point was the way that we divided up the 
  work on the document since many of the sections were dependent on each other.
  This often lead to some people on the team waiting for others to finish their
  section before they could start their own so that there wasn't conflicting
  information or text in the document. This process was slighly improved with 
  the frequent in-person work sessions, but it was still a pain point.
  
  \textbf{Nirmal Chaudhari:} While working on the environment section, intially 
  coming up with the list of components in the environment the system will 
  have to interact with as difficult. This is because of the ambuguity that 
  initially existed with what we can consider the "environment" for this system.
  This unclearity was resolved by coming up with a very high level use-case 
  scenario of what a typical person would be interacting with when using the 
  device. 
  \textbf{Sathurshan:} Many sections were dependent on others being completed
  first, which blocked some of the writing process. With the granted extension,
  several sections were delayed, leading to a time crunch toward the end. To
  address this, the team organized multiple collaborative work sessions to work
  on the SRS document together. This allowed us to exchange ideas in real time,
  resolve blockers quickly, and progress in parallel. Moving forward, during
  the project planning stage, the team should prioritize dependency related
  issues and set internal deadlines to ensure smoother progress.

  \textbf{Omar}: There was some friction when it came to sections that 
  relied on other sections being completed first. However, the team was able to
  work through these issues by holding several in-person and virtual meetings 
  to discuss the document and make progress together. This allowed us to 
  quickly resolve any blockers and ensure that everyone was on the same page.

  \item How many of your requirements were inspired by speaking to your
  client(s) or their proxies (e.g. your peers, stakeholders, potential users)?

  A lot of the requirements related to focus areas defined in our SRS document 
  were inspired by our project supervisor. He gave great insight into what major
  components need to be researched for this system to work. For example, he went 
  over the requirement of needing 4 ADC converters in our microprocessor to 
  retrieve synchronized audio input across all 4 microphones. If he didn't give 
  this insight early on, the team would have been stuck in the later stages on 
  the project with a microprocessor that will not work well for this system. 
  Furthermore, he gave good information of how the team can go about using 
  Independent Component Analysis to seperate audio into sources. He also 
  mentioned we shouldn't use deep machine learning models for audio 
  classification, since they won't be able to run on the microprocessor well. 

  
  \item Which of the courses you have taken, or are currently taking, will help
  your team to be successful with your capstone project.

  \textbf{Nirmal Chaudhari: } For this project the three courses I think will 
  enable us to be the most successful are: Signals \& Systems, Requirements 
  Engineering and Software Design 2. Signals \& Systems was important since the 
  entire project revolves around processing and analyzing audio signals in the 
  frequency domain. Requirements Engineering is important in helping us figure 
  out our requirements and ensuring throughout the entire process that what we 
  are building is the right thing. And Software Design 2 is useful in helping 
  us implement best practices into the project, thus making it sustainable 
  in the future.

  \textbf{Sathurshan:} 3MX3: signal processing, 2GA3: computer hardware
  architecture. 3RA3: software requirements, 2DA4: software design,
  3A04: software architecture, 3S03: software testing.

  \textbf{Omar}: The courses that will help us the most with this project are:
  Signals and Systems (3MX3) - This course provides a solid foundation in signal
  processing, which is crucial for our project focused on audio signal analysis.
  Concurrent Programming (3BB4) - This course teaches us the core principles
  of concurrent programming which will be essential for implementing real-time
  processing on our microcontroller.

  \textbf{Kalp:} The courses that will probably help us the most for this 
  project are 3MX3 (Signals & Systems with Dr. Mohrenschildt). This course
  taught us many of the signal processing algorithms for that we will likely 
  need to apply during our audio analysis for the system. Another good course 
  would be 3A04 (Software Architecture) that taught us how to design and 
  implement large scale software systems which will be important for our project
  as we we will be designing many modular components that work together. The 
  planning techniques from the course, specifically, will be very useful.

  \item What knowledge and skills will the team collectively need to acquire to
  successfully complete this capstone project?  Examples of possible knowledge
  to acquire include domain specific knowledge from the domain of your
  application, or software engineering knowledge, mechatronics knowledge or
  computer science knowledge.  Skills may be related to technology, or writing,
  or presentation, or team management, etc.  You should look to identify at
  least one item for each team member.

  Each memmber of the team requires the following qualfiications as contributing
  developers to the team.

  \begin{itemize}
    \item Embedded software development.
    \item Strong software design for features being implemented on any 
    microcontroller platform. 
    \item Strong testing skills.
    \item Strong debugging skills. 
  \end{itemize}

  In addition to this, since each team member is a focus area expert, they 
  require the following skills
  and competencies to carry out that role.

  \begin{itemize}\setlength\itemsep{4pt}\setlength{\leftmargini}{2em}
    \item Look through research articles, and technical evaluations to come up 
    with feasibile approaches for proposed methods. 
    \item Collaborate with other team members to discuss findings. 
    \item Maintain clear and organized documentation of sources and proposed 
    methods. 
    \item Ensure that all research and implementation choices align with project
     objectives, timeline and budgeting costs. 
    \item Based on the confirmed approach, complete the full implementation of 
    that focus in the system. 
    \item After implementation, create test cases that cover's the main 
    functionality of the feature in that focus area. 
    \item Configure github pipeline to run those tests on every PR and merge 
    into a feature branch. 
  \end{itemize}


  \item For each of the knowledge areas and skills identified in the previous
  question, what are at least two approaches to acquiring the knowledge or
  mastering the skill?  Of the identified approaches, which will each team
  member pursue, and why did they make this choice?

  \textbf{Kalp:} I think the main focus for knowledge areas to explore has to be
  around embedded software development and hardware integration. Since I've only
  done software development in industry before, I have expereince developing, 
  testing, and debugging software, but have never explored the hardware side of
  the systems. 

  \textbf{Nirmal Chaudhari:} For embedded software development and strong 
  software design, two approaches are available: (1) reviewing processor 
  documentation and (2) practicing and test building small programs on the 
  board to see how it targets key peripherals work. I will focus on the second 
  approach since practical experience seems more important. For testing and 
  debugging skills, the team can (1) study or use existing knowledge of 
  frameworks and (2), conduct peer reviews with other members on the team. 
  For this I would prefer the second approach since working with the team 
  on real bugs would help me grow as a developer to see how others resolve bugs.
  For research and technical evaluation, the knowledge can be gained by (1) 
  reading research papers existing in the focus area and (2) contacting domain 
  experts like mvm for insights. I plan to start with the first option, 
  since academic resources provides structure that can be referenced to later on.
 
  \textbf{Sathurshan:} I plan to focus on gaining knowledge in embedded software
  development, as it is something I want to specialize in the next few years.

  \textbf{Omar}: I believe the best way to learn any skill is by doing it and
  struggling through problems. Each microcontroller platform has its own quirks 
  and tools. I plan to gain practical experience by developing smaller 
  projects on the STM32 platform, which will help me understand its architecture.
  Each problem in our project can be subdivided into smaller projects, which 
  will help me learn as I go. Additionally, I will refer to the STM32 
  documentation and online tutorials to supplement my learning.

\end{enumerate}
