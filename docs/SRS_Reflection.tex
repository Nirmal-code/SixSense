\begin{enumerate}
  \item What went well while writing this deliverable? 
  \item What pain points did you experience during this deliverable, and how did
  you resolve them?
  \item How many of your requirements were inspired by speaking to your
  client(s) or their proxies (e.g. your peers, stakeholders, potential users)?

  A lot of the requirements related to focus areas defined in our SRS document were inspired
  by our project supervisor. He gave great insight into what major components need to be researched
  for this system to work. For example, he went over the requirement of needing 4 ADC converters in 
  our microprocessor to retrieve synchronized audio input across all 4 microphones. If he didn't give
  this insight early on, the team would have been stuck in the later stages on the project with a
  microprocessor that will not work well for this system. Furthermore, he gave good information
  of how the team can go about using Independent Component Analysis to seperate audio into sources.
  He also mentioned we shouldn't use deep machine learning models for audio classification, 
  since they won't be able to run on the microprocessor well. 

  
  \item Which of the courses you have taken, or are currently taking, will help
  your team to be successful with your capstone project.

  \item What knowledge and skills will the team collectively need to acquire to
  successfully complete this capstone project?  Examples of possible knowledge
  to acquire include domain specific knowledge from the domain of your
  application, or software engineering knowledge, mechatronics knowledge or
  computer science knowledge.  Skills may be related to technology, or writing,
  or presentation, or team management, etc.  You should look to identify at
  least one item for each team member.

  Each memmber of the team requires the following qualfiications as contributing developers to the team.

  \begin{itemize}
    \item Embedded software development.
    \item Strong software design for features being implemented on any microcontroller platform. 
    \item Strong testing skills.
    \item Strong debugging skills. 
  \end{itemize}

  In addition to this, since each team member is a focus area expert, they require the following skills
  and competencies to carry out that role.

  \begin{itemize}\setlength\itemsep{4pt}\setlength{\leftmargini}{2em}
    \item Look through research articles, and technical evaluations to come up with feasibile approaches for proposed methods. 
    \item Collaborate with other team members to discuss findings. 
    \item Maintain clear and organized documentation of sources and proposed methods. 
    \item Ensure that all research and implementation choices align with project objectives, timeline and budgeting costs. 
    \item Based on the confirmed approach, complete the full implementation of that focus in the system. 
    \item After implementation, create test cases that cover's the main functionality of the feature in that focus area. 
    \item Configure github pipeline to run those tests on every PR and merge into a feature branch. 
  \end{itemize}


  \item For each of the knowledge areas and skills identified in the previous
  question, what are at least two approaches to acquiring the knowledge or
  mastering the skill?  Of the identified approaches, which will each team
  member pursue, and why did they make this choice?
\end{enumerate}
